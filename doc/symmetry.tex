\documentclass{article}
\usepackage{amsmath}
\usepackage{amssymb}

\oddsidemargin 0.0cm
\textwidth 7.0in 
\textheight 9.0in 
\topmargin 0.0in 

\newcommand{\up}{\ensuremath{\uparrow}}
\newcommand{\down}{\ensuremath{\downarrow}}
\newcommand{\bra}[1]{\ensuremath{\langle#1|}}
\newcommand{\ket}[1]{\ensuremath{|#1\rangle}}
\newcommand{\melem}[3]{\ensuremath{\langle #1|#2|#3\rangle}}
\newcommand{\aver}[1]{\ensuremath{\langle#1\rangle}}
\newcommand{\bc}{\ensuremath{\bar c}}
\newcommand{\hc}{\ensuremath{\hat c}}
\newcommand{\hcd}{\ensuremath{\hat c^\dagger}}
\newcommand{\bC}{\ensuremath{\bar C}}
 
\DeclareMathOperator*{\sgn}{sgn}
\DeclareMathOperator{\Sp}{Sp}
\DeclareMathOperator*{\Tr}{Tr}

\begin{document}

\section{A short note on symmetry analysis}

Consider a general Hamiltonian of a many-body fermion system with $N$ fermionic modes and
pair interactions:
\begin{equation}\label{Hamiltonian}
    \hat H = \sum_{\alpha\beta} h_{\alpha\beta}\hcd_\alpha \hc_\beta +
        \frac{1}{2}\sum_{\alpha\beta\gamma\delta} U_{\alpha\beta\gamma\delta}
        \hcd_\alpha \hcd_\beta \hc_\gamma \hc_\delta
\end{equation}
This Hamiltonian corresponds to the following action:    
\begin{equation}\label{action}
    S = -\sum_{\omega_n}\sum_{\alpha\beta}
        \bc_{\alpha\omega_n}(i\omega_n\delta_{\alpha\beta}-h_{\alpha\beta})c_{\beta\omega_n} +
        \frac{1}{2}\int_0^\beta d\tau
        \sum_{\alpha\beta\gamma\delta}U_{\alpha\beta\gamma\delta}
        \bc_{\alpha}(\tau)\bc_{\beta}(\tau)c_{\gamma}(\tau)c_\delta(\tau)
\end{equation}

\paragraph{Theorem 1.} If there exists a permutation $N\times N$ matrix $P$ obeying additional
conditions $P_{lj}=1$ and $P_{ik}=1$, such that
\begin{equation}\label{theorem1:eq_h}
    h_{\alpha\beta} =
        \sum_{\alpha'\beta'}P_{\alpha\alpha'} h_{\alpha'\beta'} P_{\beta'\beta},
\end{equation}
\begin{equation}\label{theorem1:eq_U}
    U_{\alpha\beta\gamma\delta} = \sum_{\alpha'\beta'\gamma'\delta'}
        P_{\alpha\alpha'}P_{\beta\beta'}
        U_{\alpha'\beta'\gamma'\delta'}
        P_{\gamma'\gamma}P_{\delta'\delta},
\end{equation}
then
\begin{equation}\label{theorem1:result}
    g_{ij}(\omega_n) = g_{kl}(\omega_n).
\end{equation}

\paragraph{Proof.} By definition $g_{ij}(\omega_n)$ is
\begin{equation}
    g_{ij}(\omega_n) = -\frac{\int\mathcal{D}[\bc,c]c_{i\omega_n}\bc_{j\omega_n}e^{-S}}
        {\int\mathcal{D}[\bc,c]e^{-S}}
\end{equation}

Let's perform a change of variables which is the permutation (relabeling) of creation and annihilation
operators defined by the matrix $P$:
\begin{equation}
        \bc_{\alpha\omega_n} = \sum_{\alpha'}\bC_{\alpha'\omega_n}P_{\alpha'\alpha},\quad
        c_{\beta\omega_n} = \sum_{\beta'}P_{\beta\beta'}C_{\beta'\omega_n}
\end{equation}

In terms of the new variables the definition of a Green's function reads:
\begin{equation}\label{g_def}
    g_{ij}(\omega_n) = -\frac{\int\mathcal{D}[\bC,C]C_{k\omega_n}\bC_{l\omega_n}e^{-\tilde S}}
        {\int\mathcal{D}[\bC,C]e^{-\tilde S}}
\end{equation}
with the new action
\begin{multline}\label{action_new}
    \tilde S = -\sum_{\omega_n}\sum_{\alpha\beta}\sum_{\alpha'\beta'}
        \bC_{\alpha'\omega_n}P_{\alpha'\alpha}
            (i\omega_n\delta_{\alpha\beta}-h_{\alpha\beta})
        P_{\beta\beta'}C_{\beta'\omega_n} +\\+
        \frac{1}{2}\int_0^\beta d\tau
        \sum_{\alpha\beta\gamma\delta}
        \sum_{\alpha'\beta'\gamma'\delta'}
        \bC_{\alpha'\omega_n}\bC_{\beta'\omega_n}
        P_{\alpha'\alpha}(\tau)P_{\beta'\beta}(\tau)
        U_{\alpha\beta\gamma\delta}
        P_{\gamma\gamma'}P_{\delta\delta'}
        C_{\gamma'\omega_n}(\tau)C_{\delta'\omega_n}(\tau)
\end{multline}

Permutation matrices are unitary, thus $\sum_{\alpha\beta} P_{\alpha'\alpha}\delta_{\alpha\beta}
P_{\beta\beta'} = \sum_{\alpha} P_{\alpha'\alpha}P_{\alpha\beta'} = \delta_{\alpha'\beta'}$.
Taking into account conditions \ref{theorem1:eq_h} and \ref{theorem1:eq_U}, we find that the
action $\tilde S$ as a function of $\bC,C$ coincides with $S$ as a function of $\bc,c$. It means
that the RHS of definition \ref{g_def} is actually a definition of $g_{kl}(\omega_n)$. \textbf{Q.E.D.}

\vspace{5mm}
\textbf{TODO:} Is there an efficient algorithm that allows to check the existence of $P$?

\end{document}
