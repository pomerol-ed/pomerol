\documentclass{article}
\usepackage{amsmath}
\usepackage{amssymb}
\usepackage[dvips]{graphicx}

\oddsidemargin 0.0cm
\textwidth 7.0in 
\textheight 9.0in 
\topmargin 0.0in 

\DeclareMathOperator*{\sgn}{sgn}
\DeclareMathOperator*{\Tr}{Tr}

\newcommand{\melem}[3]{\langle#1|#2|#3\rangle}

\begin{document}

\section{One-particle fermion Green's function for a Hamiltonian system}

A thermal Green's function for a Hamiltonian system of fermions in the Matsubara representation:
\begin{eqnarray*}
    g_{12}(\omega) \equiv -\langle c_{\omega 1} \bar c_{\omega 2} \rangle = 
    -\int_0^\beta e^{i\omega\tau} d\tau \Tr [\hat w\mathbb{T} \hat c_1(\tau) \hat c^+_2(0)] =\\=
        \sum_{mn}\frac{\langle n|\hat c_1|m\rangle\langle m|\hat c^+_2|n\rangle(w_m+w_n)}
            {i\omega - (E_m-E_n)}
\end{eqnarray*}
where indices $n, m$ denote Fock states of the system, $E_n$ is an energy level, $\beta$ is the inverse
temperature and
\[
    w_n \equiv \frac{e^{-\beta E_n}}{Z}, \quad Z \equiv \sum_n e^{-\beta E_n}
\]

\section{Two-particle fermion Green's function and two-particle irreducible vertex part}

For an energy-conserving combination of frequencies $(\omega_1, \omega_2; \omega_3, \omega_4 = \omega_1+\omega_2-\omega_3)$
a two-particle Green's function is defined as follows:

\begin{equation*}
    \chi_{1234}(\omega_1,\omega_2;\omega_3,\omega_4) \equiv 
        \langle c_{\omega_1 1} c_{\omega_2 2} \bar c_{\omega_3 3} \bar c_{\omega_4 4} \rangle = 
        \int_0^\beta e^{i\omega_1\tau_1 + i\omega_2\tau_2 - i\omega_3\tau_3} d\tau 
            \Tr [\hat w\mathbb{T} \hat c_1(\tau_1) \hat c_2(\tau_2) \hat c^+_3(\tau_3) \hat c^+_4(0)]
\end{equation*}

\begin{eqnarray*}
    \chi_{1234}(\omega_1,\omega_2;\omega_3,\omega_4) =
            \sum_{\Pi}\sgn(\Pi)\sum_{ijkl}
                \langle i|\hat O_{\Pi_1}|j\rangle \langle j|\hat O_{\Pi_2}|k\rangle
                \langle k|\hat O_{\Pi_3}|l\rangle \langle l|\hat c_4^+|i\rangle 
            \phi_{ijkl}(z_{\Pi_1},z_{\Pi_2},z_{\Pi_3})
\end{eqnarray*}
where a sum over $\Pi$ is a sum over all permutations of 3 indices, $\hat O = \{\hat c_1, \hat c_2, \hat c^+_3\}$,
$z = \{i\omega_1, i\omega_2, -i\omega_3\}$. The spectral kernel $\phi_{ijkl}$ has the following form:

\begin{multline*}
    \phi_{ijkl}(z_1,z_2,z_3) = \\ =
        \frac{w_i + w_l}{(z_1+E_i-E_j)(z_1+z_2+z_3+E_i-E_l)(z_3+E_k-E_l)} -
        \frac{w_j + w_k}{(z_1+E_i-E_j)(z_2+E_j-E_k)(z_3+E_k-E_l)} + \\ +
        \frac{1}{(z_1+E_i-E_j)(z_3+E_k-E_l)}
            \left[\beta w_i\delta_{z_1+z_2}\delta_{E_i-E_k} + 
            \frac{w_k-w_i}{z_1+z_2+E_i-E_k}(1-\delta_{z_1+z_2}\delta_{E_i-E_k})\right] -\\-
        \frac{1}{(z_1+E_i-E_j)(z_3+E_k-E_l)}
            \left[\beta w_j\delta_{z_2+z_3}\delta_{E_j-E_l} + 
            \frac{w_l-w_j}{z_2+z_3+E_j-E_l}(1-\delta_{z_2+z_3}\delta_{E_j-E_l})\right]
\end{multline*}

The Wick part of a two-particle Green's function:
\begin{eqnarray*}
    \chi^0_{1234}(\omega_1,\omega_2;\omega_3,\omega_4) = 
        \beta\delta_{\omega_1\omega_4}\delta_{\omega_2\omega_3}g_{14}(\omega_1)g_{23}(\omega_2) -
        \beta\delta_{\omega_1\omega_3}\delta_{\omega_2\omega_4}g_{13}(\omega_1)g_{24}(\omega_2)
\end{eqnarray*}

An irreducible vertex part:
\[
    \Gamma_{1234}(\omega_1,\omega_2;\omega_3,\omega_4) \equiv 
        \chi_{1234}(\omega_1,\omega_2;\omega_3,\omega_4) -
        \chi^{0}_{1234}(\omega_1,\omega_2;\omega_3,\omega_4)
\]

An amputated irreducible vertex part:
\[
    \gamma_{1234}(\omega_1,\omega_2;\omega_3,\omega_4) \equiv 
        \sum_{1'2'3'4'}
        (g^{-1}(\omega_1))_{11'} (g^{-1}(\omega_2))_{22'}
        \Gamma_{1'2'3'4'}(\omega_1,\omega_2;\omega_3,\omega_4)
        (g^{-1}(\omega_3))_{3'3} (g^{-1}(\omega_4))_{4'4}
\]

\section{Singular part of an irreducible vertex part.}

The singular part of an irreducible vertex part is defined as a sum of all terms proportional to $\beta$.
It can be expressed using a function $F$ (Wick-like contribution) and a function $R$ (``superconductive'' contribution).

\begin{equation*}
    \Gamma^{s}_{1234} = 
          \beta F_{1234}(\omega_1,\omega_2) \delta_{\omega_2-\omega_3} 
        - \beta F_{2134}(\omega_2,\omega_1) \delta_{\omega_1-\omega_3}
        + \beta R_{1234}(\omega_1,\omega_3) \delta_{\omega_1+\omega_2}
\end{equation*}

\begin{eqnarray*}
    F_{1234}(\omega_1,\omega_2) = \sum_{ijkl} w_i
        \left\{
            \frac{\delta_{E_i - E_k}\melem{i}{\hat c_2}{l}\melem{l}{\hat c^+_3}{k}\melem{k}{\hat c^+_4}{j}\melem{j}{\hat c_1}{i}
                    - w_k \melem{k}{\hat c_2}{l}\melem{l}{\hat c^+_3}{k}\melem{i}{\hat c^+_4}{j}\melem{j}{\hat c_1}{i}
                }
                {(i\omega_1-(E_i-E_j))(i\omega_2-(E_l-E_k))} \right. +\\+ \left.
            \frac{\delta_{E_i - E_k}\melem{i}{\hat c^+_3}{l}\melem{l}{\hat c_2}{k}\melem{k}{\hat c^+_4}{j}\melem{j}{\hat c_1}{i}
                    - w_k \melem{k}{\hat c^+_3}{l}\melem{l}{\hat c_2}{k}\melem{i}{\hat c^+_4}{j}\melem{j}{\hat c_1}{i}
                }
                {(i\omega_1-(E_i-E_j))(i\omega_2-(E_k-E_l))} \right. +\\+ \left.
            \frac{\delta_{E_i - E_k}\melem{i}{\hat c_1}{j}\melem{j}{\hat c^+_4}{k}\melem{k}{\hat c_2}{l}\melem{l}{\hat c^+_3}{i}
                    - w_k \melem{i}{\hat c_1}{j}\melem{j}{\hat c^+_4}{i}\melem{k}{\hat c_2}{l}\melem{l}{\hat c^+_3}{k}
                }
                {(i\omega_1-(E_j-E_i))(i\omega_2-(E_l-E_k))} \right. +\\+ \left.
            \frac{\delta_{E_i - E_k}\melem{i}{\hat c_1}{j}\melem{j}{\hat c^+_4}{k}\melem{k}{\hat c^+_3}{l}\melem{l}{\hat c_2}{i}
                    - w_k \melem{i}{\hat c_1}{j}\melem{j}{\hat c^+_4}{i}\melem{k}{\hat c^+_3}{l}\melem{l}{\hat c_2}{k}
                }
                {(i\omega_1-(E_j-E_i))(i\omega_2-(E_k-E_l))}
        \right\}
\end{eqnarray*}

\begin{eqnarray*}
    R_{1234}(\omega_1,\omega_3) = -\sum_{ijkl}\delta_{E_i - E_k} w_i
        \left\{
            \frac{\melem{i}{\hat c_1}{j}\melem{j}{\hat c_2}{k}\melem{k}{\hat c^+_3}{l}\melem{l}{\hat c^+_4}{i}}
                {(i\omega_1-(E_j-E_i))(i\omega_3-(E_k-E_l))}
          + \frac{\melem{i}{\hat c_2}{j}\melem{j}{\hat c_1}{k}\melem{k}{\hat c^+_3}{l}\melem{l}{\hat c^+_4}{i}}
                {(i\omega_1-(E_i-E_j))(i\omega_3-(E_k-E_l))} \right. +\\+ \left.
            \frac{\melem{k}{\hat c_1}{j}\melem{j}{\hat c_2}{i}\melem{i}{\hat c^+_4}{l}\melem{l}{\hat c^+_3}{k}}
                {(i\omega_1-(E_j-E_i))(i\omega_3-(E_l-E_k))}
          + \frac{\melem{k}{\hat c_2}{j}\melem{j}{\hat c_1}{i}\melem{i}{\hat c^+_4}{l}\melem{l}{\hat c^+_3}{k}}
                {(i\omega_1-(E_i-E_j))(i\omega_3-(E_l-E_k))}
        \right\}
\end{eqnarray*}    

{\bf $\beta\to\infty$ limit.}
Weight $w_i$ goes to $1/g$ in the limit of low temperatures, if $i$ is a component of a $g$-fold ground state, and vanishes for excited states.
Thus a summation over $i$ and $k$ in the formulae above includes only components of the ground state:
\[
    \sum_{ijkl} w_i\delta_{E_i-E_k} \mapsto \frac{1}{g}\sum_{ik\in \{|gs\rangle\}}\sum_{jl}
\]

\[
    \sum_{ijkl} w_i w_k \mapsto \frac{1}{g^2}\sum_{ik\in \{|gs\rangle\}}\sum_{jl}
\]

If the ground state is not degenerate ($g=1$), then $\beta F_{1234}(\omega_1,\omega_2) \to 0$ as $\beta\to\infty$.
It is easy to prove taking the limit for this special case:
\[
    w_i \to \delta_{i,gs};
    \qquad \sum_{ijkl} w_i\delta_{E_i,E_k},\ \sum_{ijkl} w_i w_k \mapsto \sum_{jl};
    \qquad E_i, E_k \mapsto E_{gs}; \qquad |i\rangle, |k\rangle \mapsto |gs\rangle
\]

\end{document}
